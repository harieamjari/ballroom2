\documentclass{article}


\title{Ballroom Group 2}

\author{Other 17 people, STEM-A 12}

\usepackage{titlesec}
\titleformat{\chapter}[display]
{\bfseries\filcenter}
{\huge\chaptername~\thechapter} %
{6ex}
{\Huge\thispagestyle{empty}}

\begin{document}
\maketitle

\abstract
\centerline{(To be removed by editor)}
Recipients of this candidate are invited to submit their comments, suggestions and recommendations
for improvement of which they are aware to provide constructive criticism. Critics in this way
are valuable and could help in formulating next future releases.

\tableofcontents
\clearpage

\section{Fundamentals}
 Proper body posture, body alignment, and body tone. 
 Essential to comfortable and effective dancing is good posture, achieved through the proper alignment of the various body parts in correct relative position with one another. Proper body alignment and posture are essential to dancers as it makes them appear more elegant and confident. It also improves overall balance and body control.
 . Detailed description and demonstration of the dance frame and dance hold.
 The frame is the way the dancers hold their hands, arms, shoulders, neck, head and upper torso. In ballroom dancing, frame is the way the dancers’ upper bodies are held when in dancing position. A good frame helps with balance and movement and also produces a good appearance of the dance couple.
 The partners are holding onto each other and ready to begin. The closed dance hold is one of the basic positions to start any dance step. Variations of this position can show initiative and the ability to remain in control of the original position. Each partner can step back just a little so that the shoes do not touch, and create a dance style that will astonish onlookers. The closed dance hold position can be performed without the full frontal body contact. 

The importance of using all body joints and proper footwork
 In ballroom dancing, the requirement for good technique and precision means correct use of the legs and feet – your footwork. Everything a dancer does – rising, leaping, slowness, quickness, swaying and action -are connected with and credited to the body joint and good/proper footwork.
\begin{enumerate}
\item Triple step - Triple steps are popular in swing dancing. The Triple Step is a three step sequence taken on two beats of music. If the first step of the triple step is taken on count 1, the second step is taken on the half beat between counts 1\&2, and the third step is taken on count 2. The step timing is often called out as 1\&2. Usually the triple step is two quick steps and one slow, called out as "quick-quick-slow", or, using numbers, as "one-and-two.”
 
\item Rock Step - Here we see a sequence of two steps called a rock step. The step timing is usually slow-slow.
 
 \item  Basic of East Coast Swing - The combination of two triple steps and a rock step form the basic step of triple timing swing or the East Coast Swing. The step timing is usually called out as 1\&2, 3\&4, 5,6.
 
 \item  Ball-Change - Here we see a sequence of two steps called a ball-change. Weight on the ball of the foot is changed to the other foot.
 
 \item  Kick Ball Change - A popular swing dance step is the kick-ball-change step, which can be used to replace the rock step. The timing is usually 1\&2.
 
 \item  The Basic Step of the Carolina Shag - The combination of two triple steps and a kick ball change can be used in triple timing swing dances such as the Carolina shag.
 
 \item  Coaster Step - The coaster step is usually a back-together-forward triple step danced to the timing of 1\&2 or quick, quick, slow.
 
 \item  Sailor Step - The sailor step has a side to side look. It is also a triple step danced to the timing of 1\&2. The step is accomplished by leaning in the opposite direction of the crossing foot.
 
 \item  Anchor Step - The anchor step is a stationary triple step danced in third foot position to the timing of 1\&2. It is popular in the west coast swing.
 
 \item  Grapevine - The grapevine is a continuous traveling step pattern to the side usually with alternating crosses behind and in front of the supporting foot.
 
 \item  Lock Step – The lock step is usually danced to triple step timing. During the step, the lower part of the legs cross such that the back leg becomes locked behind the leading leg until the leading leg moves forward. The lock step is often used in the triple step of the cha cha cha
\end{enumerate}
\section{Characteristics}
(Body to be provided by members of the group)


\section{History}
(Body to be provided by members of the group)

\section{Nature of dance}
Dance is one of the most beautiful forms of art that has grown in leaps and bounds. Dance is no longer just a hobby. it is additionally one of the foremost profitable callings as well. Other than, move treatment is exceptionally much in fashion these days essentially since the involvement of moving helps a person to recuperate from inside. Move may be a shape of expression that makes a difference a individual bring forward who they are, and what they're enthusiastic almost.
\end{document}
